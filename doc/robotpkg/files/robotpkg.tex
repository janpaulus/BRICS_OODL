% $LAAS: robotpkg.tex 2009/03/04 16:49:53 mallet $
%
% Copyright (c) 2009 LAAS/CNRS
% All rights reserved.
%
% Permission to use, copy, modify, and distribute this software for any purpose
% with or without   fee is hereby granted, provided   that the above  copyright
% notice and this permission notice appear in all copies.
%
% THE SOFTWARE IS PROVIDED "AS IS" AND THE AUTHOR DISCLAIMS ALL WARRANTIES WITH
% REGARD TO THIS  SOFTWARE INCLUDING ALL  IMPLIED WARRANTIES OF MERCHANTABILITY
% AND FITNESS. IN NO EVENT SHALL THE AUTHOR  BE LIABLE FOR ANY SPECIAL, DIRECT,
% INDIRECT, OR CONSEQUENTIAL DAMAGES OR  ANY DAMAGES WHATSOEVER RESULTING  FROM
% LOSS OF USE, DATA OR PROFITS, WHETHER IN AN ACTION OF CONTRACT, NEGLIGENCE OR
% OTHER TORTIOUS ACTION,   ARISING OUT OF OR IN    CONNECTION WITH THE USE   OR
% PERFORMANCE OF THIS SOFTWARE.
%
%                                             Anthony Mallet on Sat Jan 10 2009
%

\documentclass[a4paper,11pt]{book}
\newif\iftth\iftth
   % $LAAS: robotpkg.sty 2009/02/01 20:07:09 tho $
%
% Copyright (c) 2009 LAAS/CNRS
% All rights reserved.
%
% Permission to use, copy, modify, and distribute this software for any purpose
% with or without   fee is hereby granted, provided   that the above  copyright
% notice and this permission notice appear in all copies.
%
% THE SOFTWARE IS PROVIDED "AS IS" AND THE AUTHOR DISCLAIMS ALL WARRANTIES WITH
% REGARD TO THIS  SOFTWARE INCLUDING ALL  IMPLIED WARRANTIES OF MERCHANTABILITY
% AND FITNESS. IN NO EVENT SHALL THE AUTHOR  BE LIABLE FOR ANY SPECIAL, DIRECT,
% INDIRECT, OR CONSEQUENTIAL DAMAGES OR  ANY DAMAGES WHATSOEVER RESULTING  FROM
% LOSS OF USE, DATA OR PROFITS, WHETHER IN AN ACTION OF CONTRACT, NEGLIGENCE OR
% OTHER TORTIOUS ACTION,   ARISING OUT OF OR IN    CONNECTION WITH THE USE   OR
% PERFORMANCE OF THIS SOFTWARE.
%
%                                             Anthony Mallet on Fri Jan 16 2009
%

\NeedsTeXFormat{LaTeX2e}
\ProvidesPackage{robotpkg}[2009/01/10 A guide to robotpkg LaTeX style]

\RequirePackage{color}
\definecolor{color@link}{rgb}{.4,.02,.01}

\RequirePackage{type1cm}
\RequirePackage{ifpdf}
\RequirePackage[pdftex]{hyperref}

\hypersetup{plainpages=false}
\hypersetup{colorlinks=true,linkcolor=color@link,urlcolor=color@link}

\ifpdf\pdfinfo{
   /Author (Anthony Mallet)
   /Title  (A guide to robotpkg)
   /CreationDate (D:20090110000000)
   /Subject (robotpkg)
   /Keywords (robotic software package)
}\fi


% --- Sections -------------------------------------------------------------

% chapters
\def\chapter{%
   \if@openright\newpage{\pagestyle{empty}\cleardoublepage}\else\clearpage\fi%
   \thispagestyle{empty}%
   \global\@topnum\z@%
   \@afterindentfalse%
   \secdef\@chapter\@schapter%
}

\def\@makechapterhead#1{%
   \vrule height 20pt width 0pt\par{%
      \parindent\z@%
      \raggedleft\normalfont%
      \ifnum\c@secnumdepth>\m@ne%
	 \if@mainmatter%
            \fontsize{40}{40}\fontfamily{ptm}\bfseries%
	    \thechapter\par\vrule height 0pt width 0pt\par%
	 \fi%
      \fi%
      \interlinepenalty\@M\fontsize{40}{45}\bfseries%
      \parbox{\textwidth}{\raggedleft#1}\par%
      \nobreak\vskip\evensidemargin%
   }%
}

\def\@makeschapterhead#1{
   \vrule height 20pt width 0pt\par{%
      \parindent\z@
      \raggedleft\normalfont
      \interlinepenalty\@M\fontsize{40}{45}\fontfamily{ptm}\bfseries
      \parbox{\textwidth}{\raggedleft#1}\par
      \nobreak\vskip\evensidemargin
   }
}


% output a coma if not already looking at a punctuation sign
%
\def\singlepunct{\@ifnextchar.\relax{\@ifnextchar,\relax{, }}}

\def\xref#1#2{#2, \nameref{#1}\singlepunct}
\def\doublehline{\\\hline\hline}

\else
   \usepackage[T1]{fontenc}
   \usepackage{robotpkg}
\fi

\title{A guide to robotpkg}
\author{
   Anthony Mallet --- {\tt anthony.mallet@laas.fr}\\[1em]
   Copyright 2006-2009 \copyright LAAS/CNRS
}
\date{\today}

\def\robotpkg{{\tt robotpkg} }

\begin{document} % ---------------------------------------------------------

\frontmatter
\maketitle
\tableofcontents
\mainmatter

\chapter{Introduction}
\label{chapter:introduction}
% $LAAS: introduction.tex 2010/01/29 22:41:15 tho $
%
% Copyright (c) 2009-2010 LAAS/CNRS
% All rights reserved.
%
% Permission to use, copy, modify, and distribute this software for any purpose
% with or without   fee is hereby granted, provided   that the above  copyright
% notice and this permission notice appear in all copies.
%
% THE SOFTWARE IS PROVIDED "AS IS" AND THE AUTHOR DISCLAIMS ALL WARRANTIES WITH
% REGARD TO THIS  SOFTWARE INCLUDING ALL  IMPLIED WARRANTIES OF MERCHANTABILITY
% AND FITNESS. IN NO EVENT SHALL THE AUTHOR  BE LIABLE FOR ANY SPECIAL, DIRECT,
% INDIRECT, OR CONSEQUENTIAL DAMAGES OR  ANY DAMAGES WHATSOEVER RESULTING  FROM
% LOSS OF USE, DATA OR PROFITS, WHETHER IN AN ACTION OF CONTRACT, NEGLIGENCE OR
% OTHER TORTIOUS ACTION,   ARISING OUT OF OR IN    CONNECTION WITH THE USE   OR
% PERFORMANCE OF THIS SOFTWARE.
%
%                                             Anthony Mallet on Sat Jan 10 2009
%

\section{What is robotpkg?} % ----------------------------------------------

The robotics research  community has always been developing  a lot of software,
in order  to illustrate theoretical concepts and  validate algorithms  on board
real robots.  A great amount of this software was  made freely available to the
community, especially for Unix-based systems,  and is usually available in form
of the source code. Therefore, before such software can be used, it needs to be
configured to  the local system, compiled and  installed.  This is exactly what
The Robotics Packages Collection (robotpkg) does.  robotpkg also has some basic
commands  to handle binary packages,  so that not  every user  has to build the
packages for himself, which is a time-costly, cumbersome and error-prone task.

The robotpkg project was initiated in the \href{http://www.laas.fr/}{Laboratory
for Analysis and Architecture of  Systems} (CNRS/LAAS), France.  The motivation
was, on the one hand,  to ease the software   maintenance tasks for the  robots
that are used there.   On the other  hand, roboticists at CNRS/LAAS have always
fostered  an  open-source  development   model  for   the   software they  were
developing.  In order to  help people  working with the  laboratory to  get the
LAAS software  running outside the laboratory,  a package management system was
necessary.

Although  robotpkg was an  innovative   project in  the robotics community  (it
started in 2006), a lot of general-purpose software packages management systems
were readily available at this time for  a great variety of Unix-based systems.
The main requirements that we wanted  robotpkg to fullfill  were listed and the
best existing package management system  was chosen as  a starting point.   The
biggest requirement was the  capacity of the system to  adapt to the  nature of
the robotic software,  being available mostly in form  of source code  only (no
binary packages),  with unfrequent stable  releases.  robotpkg had thus to deal
mostly with  source code  and automate the  compilation of  the  packages.  The
system chosen  as a starting  point was \href{http://www.pkgsrc.org}{The NetBSD
Packages  Collection} (pkgsrc).  robotpkg  can be considered as  a fork of this
project and  it is still very similar  to pkgsrc in  many points, although some
simplifications were made in order to provide  a tool geared toward people that
are not computer scientists but roboticists.

Due to its  origins, robotpkg provides many packages  developed at LAAS.  It is
however not  limited to such  packages and contains, in  fact, quite some other
software useful to  roboticists.  Of  course, robotpkg  is  not meant to  be  a
general purpose  packaging system   (although  there  would  be   no  technical
restriction to this) and will never  contain widely available packages that can
be found  on  any modern  Unix  distribution. Yet, robotpkg currently  contains
roughly one hundred and fifty packages, including:

\begin{itemize}
   \item architecture/genom - The LAAS Generator of Robotic Components

   \item architecture/openhrp - The Open Architecture Humanoid Robotics
   Platform from AIST, Japan

   \item architecture/openrtm - The robotic distributed middleware from AIST, Japan

   \item architecture/yarp - The ``other'', yet famous, robot platform

   \item ...just to name a few.
\end{itemize}


\section{Why robotpkg?} % --------------------------------------------------

robotpkg provides the following key features:

\begin{itemize}

   \item Easy building of software  from  source as well   as the creation  and
   installation of binary packages. The source and latest patches are retrieved
   from a master download site, checksum verified, then built on your system.

   \item All  packages are installed in a  consistent directory tree, including
   binaries, libraries, man pages and other documentation.

   \item  Package dependencies, including  when performing package updates, are
   handled automatically.

   \item The installation prefix, acceptable  software licenses and  build-time
   options  for a large  number of packages  are all set  in  a simple, central
   configuration file.

   \item The  entire framework source  (not including the  package distribution
   files themselves) is freely available under a BSD license, so you may extend
   and adapt robotpkg to your needs, like robotpkg was adapted from pkgsrc.

\end{itemize}

\section{Supported platforms} % --------------------------------------------

robotpkg consists of  a   source distribution. After retrieving    the required
source, you can be up and running with robotpkg in just minutes!

robotpkg  does not have much requirements  by itself and it  can work on a wide
variety of systems  as  long as they   provide a  GNU-make utility, a   working
C-compiler and a small, reasonably standard subset  of Unix commands (like sed,
awk, find,  grep ...).  However, individual packages  might have their specific
requirements.  The   following platforms  have been  reported  to  be supported
reasonably well:

\begin{center}\begin{tabular}{|c|c|}
\hline
Platform & Version
\doublehline
Fedora & 5 -- 12\\
Ubuntu & 7.10 -- 9.04\\
Debian & 5.03\\
CentOS & 5\\
NetBSD & 4 -- 5\\
Darwin & Preliminary support\\
\hline
\end{tabular}\end{center}


\section{Overview} % -------------------------------------------------------

This document is divided  into three parts.  \xref{chapter:user}{The first one}
describes how  one  can  use  one of   the  packages  in the  Robotics  Package
Collection, either  by installing a precompiled binary  package, or by building
one's own  copy  using  robotpkg.   \xref{chapter:developer}{The  second  part}
explains how  to prepare a package so  it can be  easily  built by  other users
without     knowing     about     the     package's    building        details.
\xref{chapter:internal}{The   third part} is  intended for  those   who want to
understand how robotpkg is implemented.


\section{Terminology} % ----------------------------------------------------

Here is a description of all the terminology used within this document.

\begin{description}
   \item[Package] A set of files and building instructions that describe what's
   necessary to build a certain piece  of software using robotpkg. Packages are
   traditionally stored under {\tt /opt/robotpkg}.

   \item[robotpkg]  This is  the The Robotics   Package Collection.  It handles
   building (compiling), installing, and removing of packages.

   \item[Distfile] This  term describes the file  or files that are provided by
   the author of the piece of software to distribute  his work. All the changes
   necessary to  build are reflected  in the corresponding package. Usually the
   distfile is in  the form of a  compressed  tar-archive, but other  types are
   possible,     too.    Distfiles   are      usually   stored    below    {\tt
   /opt/robotpkg/distfiles}.

   \item[Precompiled/binary package] A set of binaries built with robotpkg from
   a distfile  and stuffed together in a  single {\tt .tgz} file   so it can be
   installed  on machines of the same  machine architecture without the need to
   recompile. Packages are usually generated in {\tt /opt/robotpkg/packages}.

   Sometimes, this is  referred to by the  term ``package''  too, especially in
   the context of precompiled packages.

   \item[Program]  The  piece  of  software to  be  installed  which  will   be
   constructed from all the files in the distfile by the actions defined in the
   corresponding package.

\end{description}


\section{Roles involved in robotpkg} % -------------------------------------

\begin{description}
   \item[robotpkg users] The  robotpkg users  are people  who  use the packages
   provided by robotpkg.  Typically they are student  working  in robotics. The
   usage  of the software  that is {\em inside} the  packages is not covered by
   the robotpkg guide.

   There are two  kinds of robotpkg users:  Some only want to install pre-built
   binary packages.  Others build the robotpkg packages from source, either for
   installing them  directly or for building binary   packages themselves.  For
   robotpkg users, \xref{chapter:user}{Part~\ref{chapter:user}}  should provide
   all necessary documentation.

   \item[package  maintainers]   A   package maintainer  creates  packages   as
   described in \xref{chapter:developer}{Part~\ref{chapter:developer}}.

   \item[infrastructure  developers]  These people are    involved in all those
   files that live  in the {\tt mk/} directory   and below.  Only  these people
   should             need          to               read               through
   \xref{chapter:internal}{Part~\ref{chapter:internal}}, though others might be
   curious, too.

\end{description}


\section{Typography} % -----------------------------------------------------

When giving examples for  commands,  shell prompts  are  used  to show if   the
command  should/can be issued  as  root, or if  ``normal''  user privileges are
sufficient. We use  a {\tt \#}  for  root's shell  prompt, and  a {\tt \%}  for
users' shell prompt, assuming they use the C-shell or tcsh.


\chapter{The robotpkg user's guide}
\label{chapter:user}

Basically, there are two ways of using robotpkg.  The  first is to only install
the  package tools and to  use binary packages that  someone else has prepared.
The second way is  to install the  programs from source. Then  you are  able to
build your own packages,  and you can  still use  binary packages from  someone
else. Sections in this document will detail both approaches where appropriate.

% $LAAS: getting.tex 2010/06/23 15:03:35 mallet $
%
% Copyright (c) 2009-2010 LAAS/CNRS
% All rights reserved.
%
% Permission to use, copy, modify, and distribute this software for any purpose
% with or without   fee is hereby granted, provided   that the above  copyright
% notice and this permission notice appear in all copies.
%
% THE SOFTWARE IS PROVIDED "AS IS" AND THE AUTHOR DISCLAIMS ALL WARRANTIES WITH
% REGARD TO THIS  SOFTWARE INCLUDING ALL  IMPLIED WARRANTIES OF MERCHANTABILITY
% AND FITNESS. IN NO EVENT SHALL THE AUTHOR  BE LIABLE FOR ANY SPECIAL, DIRECT,
% INDIRECT, OR CONSEQUENTIAL DAMAGES OR  ANY DAMAGES WHATSOEVER RESULTING  FROM
% LOSS OF USE, DATA OR PROFITS, WHETHER IN AN ACTION OF CONTRACT, NEGLIGENCE OR
% OTHER TORTIOUS ACTION,   ARISING OUT OF OR IN    CONNECTION WITH THE USE   OR
% PERFORMANCE OF THIS SOFTWARE.
%
%                                             Anthony Mallet on Sat Jan 10 2009
%


\section{Where to get robotpkg and how to keep it up-to-date} % ------------
\label{section:getting}

Before you download and extract the files, you need to decide where you want to
extract them and  where you want robotpkg to install  packages.  By defaut, the
{\tt  /opt/openrobots} directory is  used. In  the rest  of this  document, the
installation path is called the {\em prefix}.

{\tt robotpkg}  will {\em never}  require administration privileges  by itself.
We thus recommend that you do not install or run robotpkg as the root user.  If
something ever goes really  wrong, it might go less wrong if  it is not running
as root. If you want to  install to the default location {\tt /opt/openrobots},
we recommend that you create this directory owned by a regular user.

Creating or using {\tt /opt/openrobots} typically requires administration ({\em
a.k.a.}  ``{\tt root}'') privileges.  If  you don't have such privileges (or if
you want  to install to a different  location), you have to  unpack the sources
and  install the  binary packages  in  another prefix.  If you  don't have  any
special administration  rights on the target  machine, a safe bet  is to choose
the {\tt  \${HOME}/openrobots} location, as  the {\tt \${HOME}}  directory will
always be writable by yourself.

Any prefix  will work, but please  note that you should  choose an installation
path which is dedicated to robotpkg packages and not shared with other programs
(e.g., we do  not recommend to use  a prefix of {\tt /usr}).   Also, you should
not try to add any of your  own files or directories (such as {\tt src/}) below
the prefix  tree.  This  will prevent possible  conflicts between  programs and
other files  installed by the  package system and  whatever else may  have been
installed there.

Finally,  the  installation  path   shall  not  contain  white-space  or  other
characters that are interpreted specially by the shell and some other programs:
use only letters, digits, underscores and dashes.

The rest of this document will  assume that you are using {\tt /opt/openrobots}
as the prefix. You should adapt this path to whatever prefix you choosed.


\subsection{Getting the binary bootstrap kit}

At the moment, the  binary bootstrap kit is not  available. Please get the {\tt
robotpkg} sources as described in the next section.


\subsection{Getting robotpkg for source compilation}

{\tt         robotpkg}    sources   are       distributed     {\em    via}  the
\href{http://git-scm.com/}{\tt git}  software  content management system.  {\tt
git} will probably be readily available on your system but if you don't have it
installed   or if you    are  unsure  about it,   contact  your  local   system
administrator.

There are two download methods: the anonymous one and the authenticated
one:

\begin{itemize}

  \item  Anonymous download is the  recommended method if  you  don't intend to
  work on  the   robotpkg infrastructure  itself,  nor commit   any  changes or
  packages  additions  back to  the  robotpkg main repository. Furthermore, the
  possibility to send contributions via patches is still open.

  As your regular user, simply run in a shell:

\begin{verbatim}
% cd /opt/openrobots
% git clone http://softs.laas.fr/git/robots/robotpkg.git
\end{verbatim}


  \item Authenticated   download requires a  valid  login on the  main robotpkg
  repository, and will give you full commit access  to this repository.  Simply
  run the following:

\begin{verbatim}
% cd /opt/openrobots
% git clone ssh://softs.laas.fr/git/robots/robotpkg
\end{verbatim}

\end{itemize}


\subsection{Keeping robotpkg up-to-date} % ---------------------------------

{\tt robotpkg} is   a   living thing:    updates  to  the packages are     made
perdiodicaly,  problems are fixed,   enhancements  are developed\ldots If   you
downloaded the robotpkg sources via git, you  should keep it up-to-date so that
you get the most  recent packages descriptions. This is done by running {\tt
git pull} in the robotpkg source directory:

\begin{verbatim}
% cd /opt/openrobots/robotpkg
% git pull
\end{verbatim}

When you update robotpkg, the git program will only  touch those files that are
registered in the git repository. That means that any packages that you created
on your own will stay unmodified. If you change  files that are managed by git,
later updates will try to merge your changes with those that  have been done by
others. See the {\tt git-pull} manual for details.

If you want  to be informed  of package additions  and other  updates, a public
mailing    list  is   available    for   your    reading   pleasure.  Go     to
\url{https://sympa.laas.fr/sympa/info/robotpkg}    for   more  information  and
subscription.

% $LAAS: bootstrapping.tex 2010/06/23 15:04:49 mallet $
%
% Copyright (c) 2009-2010 LAAS/CNRS
% All rights reserved.
%
% Permission to use, copy, modify, and distribute this software for any purpose
% with or without   fee is hereby granted, provided   that the above  copyright
% notice and this permission notice appear in all copies.
%
% THE SOFTWARE IS PROVIDED "AS IS" AND THE AUTHOR DISCLAIMS ALL WARRANTIES WITH
% REGARD TO THIS  SOFTWARE INCLUDING ALL  IMPLIED WARRANTIES OF MERCHANTABILITY
% AND FITNESS. IN NO EVENT SHALL THE AUTHOR  BE LIABLE FOR ANY SPECIAL, DIRECT,
% INDIRECT, OR CONSEQUENTIAL DAMAGES OR  ANY DAMAGES WHATSOEVER RESULTING  FROM
% LOSS OF USE, DATA OR PROFITS, WHETHER IN AN ACTION OF CONTRACT, NEGLIGENCE OR
% OTHER TORTIOUS ACTION,   ARISING OUT OF OR IN    CONNECTION WITH THE USE   OR
% PERFORMANCE OF THIS SOFTWARE.
%
%                                             Anthony Mallet on Sun Jan 11 2009
%

\section{Bootstrapping robotpkg} % -----------------------------------------
\label{section:bootstrapping}

Once you have  downloaded the robotpkg sources  or the binary bootstrap kit  as
described  in  \xref{section:getting}{Section~\ref{section:getting}}, a minimal
set  of  the administrative package management  utilities  must be installed on
your system  before you  can  use robotpkg.   This  is  called the  ``bootstrap
phase'' and  should   be done only   once,  the very  first  time you  download
robotpkg.


\subsection{Bootstrapping via the binary kit} % ----------------------------

At the moment, the binary bootstrap kit is not available. Please bootstrap {\tt
robotpkg} as described in the next section.


\subsection{Bootstrapping from source} % -----------------------------------

You will  need a working C compiler  and the GNU-make   utility version 3.81 or
later.    If you have  extracted  the  robotpkg  archive  into  the standard {\tt
/opt/openrobots/robotpkg} location, installing the   bootstrap kit from  source
should then be as simple as:

\begin{verbatim}
% cd /opt/openrobots/robotpkg/bootstrap
% ./bootstrap
\end{verbatim}

This will  install various utilities   into {\tt /opt/openrobots/sbin}.

Should you prefer another installation path, you could use the {\tt -{}-prefix}
option to  change the default  installation prefix.  For  instance, configuring
robotpkg  to  install programs  into  the  openrobots  directory in  your  home
directory can be done like this:

\begin{verbatim}
% cd robotpkg/bootstrap
% ./bootstrap --prefix=${HOME}/openrobots
\end{verbatim}

{\bf  After the  bootstrap script  has run,  a message  indicating  the success
should be  displayed.  If  you choosed a  non-standard installation  path, read
this message carefuly}, as it contains  instructions that you have to follow in
order  to  setup your  shell  environment  correctly.   These instructions  are
described in the next section.


\subsubsection{Configuring your environment} % -----------------------------

If  you configured robotpkg,   during the bootstrap  phase,  to install to some
other location   than {\tt /opt/openrobots}, you  have   to setup manually your
shell environment so that it contains a few  variables holding the installation
path.  Assuming  you invoked bootstrap with {\tt --prefix=/path/to/openrobots},
you have two options that are compatible with each other:

\begin{itemize}
   \item Add  the directory {\tt  /path/to/openrobots/sbin}  to your {\tt PATH}
   variable. robotpkg will    then be able  to find    its administrative tools
   automatically and from that recover other configuration information. This is
   the preferred method.

   \item Create the environment variable {\tt ROBOTPKG\_BASE} and set its value
   to {\tt /path/to/openrobots}.  robotpkg will  look for this variable  first,
   so it takes precedence over the  first method.  This is  the method you have
   to choose  if  you have  configured  several instances  of robotpkg  in your
   system. This is ony useful in some circumstances and is not normally needed.
\end{itemize}

If  you  don't know  how  to setup   environment variables  permanently in your
system,  please  refer  to  your shell's  manual  or contact  your local system
administrator.


\subsubsection{The bootstrap script usage} % -------------------------------

The {\tt bootstrap} script will by default install the package administrative
tools in {\tt /opt/openrobots/sbin}, use {\tt gcc} as the C compiler and {\tt
make} as the GNU-make program. This behaviour can be fine-tuned by using the
following options:

\begin{description}
   \item[\tt   -{}-prefix <path>]   will   select the  prefix  location where
   programs will be installed in.

   \item[\tt -{}-sysconfdir <path>] defaults to {\tt <prefix>/etc}. This is the
   path to the robotpkg configuration file.  Other packages configuration files
   (if any) will also be stored in this directory.

   \item[\tt -{}-pkgdbdir  <path>] defaults to {\tt  <prefix>/var/db/pkg}. This
   is the path  to the package database  directory  where robotpkg will  do its
   internal bookkeeping.

   \item[\tt -{}-compiler <program>] defaults to {\tt gcc}.  Use this option if
   you want to use a different C compiler.

   \item[\tt -{}-make <program>] defaults to {\tt make}. Use this option if you
   want to use a different make program. This program should be compatible with
   GNU-make.

   \item[\tt -{}-help]  displays  the {\tt bootstrap} usage.  The comprehensive
   list of recognized options will be displayed.
\end{description}

% $LAAS: using.tex 2010/06/28 10:24:09 mallet $
%
% Copyright (c) 2009-2010 LAAS/CNRS
% All rights reserved.
%
% Permission to use, copy, modify, and distribute this software for any purpose
% with or without   fee is hereby granted, provided   that the above  copyright
% notice and this permission notice appear in all copies.
%
% THE SOFTWARE IS PROVIDED "AS IS" AND THE AUTHOR DISCLAIMS ALL WARRANTIES WITH
% REGARD TO THIS  SOFTWARE INCLUDING ALL  IMPLIED WARRANTIES OF MERCHANTABILITY
% AND FITNESS. IN NO EVENT SHALL THE AUTHOR  BE LIABLE FOR ANY SPECIAL, DIRECT,
% INDIRECT, OR CONSEQUENTIAL DAMAGES OR  ANY DAMAGES WHATSOEVER RESULTING  FROM
% LOSS OF USE, DATA OR PROFITS, WHETHER IN AN ACTION OF CONTRACT, NEGLIGENCE OR
% OTHER TORTIOUS ACTION,   ARISING OUT OF OR IN    CONNECTION WITH THE USE   OR
% PERFORMANCE OF THIS SOFTWARE.
%
%                                             Anthony Mallet on Sun Jan 11 2009
%

\section{Using robotpkg} % -------------------------------------------------

After obtaining \robotpkg, the  {\tt robotpkg} directory now  contains a set of
packages, organized   into  categories.  You can   browse  the online  index of
packages, or run {\tt  make index} from the {\tt  robotpkg} directory to  build
local {\tt index.html}  files for all  packages, viewable with any web  browser
such as {\tt lynx} or {\tt firefox}.


\subsection{Building packages from source} % -------------------------------

The  first step for  building  a  package  is  downloading the {\em  distfiles}
(i.e. the unmodified  source). If they have not  yet been downloaded, \robotpkg
will  fetch them automatically  and place them  in the {\tt robotpkg/distfiles}
directory.

Once the software  has  been downloaded,  any  patches will be applied  and the
package will  be compiled for  you.  This may  take some time depending on your
computer, and how many other packages the software depends on and their compile
time.

For  example,  type the following  commands  at the shell   prompt to build the
robotpkg documentation package:

\begin{verbatim}
% cd /opt/openrobots/robotpkg
% cd doc/robotpkg
% make
\end{verbatim}

The next  stage is  to  actually install the newly   compiled package onto your
system. While you   are still in  the directory  for whatever package  you  are
installing, you can do this by entering:

\begin{verbatim}
% make install
\end{verbatim}

Installing the package on your system does  not require you  to be root (except
for a few specific  packages). However, if   you bootstraped with a  prefix for
which   you  don't   have  writing   permissions,    \robotpkg   has a     {\rm
just-in-time-sudo}  feature,  which allows you to  become  {\tt  root}  for the
actual installation step.

That's it, the software should now be installed   under  the prefix  of the
packages tree --- {\tt /opt/openrobots} by default --- and setup for use.

You can now enter:

\begin{verbatim}
% make clean
\end{verbatim}

to remove the compiled files in the work  directory, as you shouldn't need them
any more. If  other packages were also  added to your system (dependencies)  to
allow your program to compile, you can also tidy these up with the command:

\begin{verbatim}
% make clean-depends
\end{verbatim}

Since  the three tasks of building,  installing and  cleaning correspond to the
typical usage of \robotpkg, a helper target doing all these tasks exists and is
called {\tt update}. Thus,  to intall a package  with a single command, you can
simply run:

\begin{verbatim}
% make update
\end{verbatim}

In addition, {\tt  make update} will  also recompile all the installed packages
that were depending on the package that you are updating. This can be quite
time consuming if you are updating a low-level package. Also, note that all
packages that depend on the package you are updating will be deinstalled
first and unavailable in your system until all packages are recompiled and
reinstalled.

%
%    <para>Some packages look in &mk.conf; to
%    alter some configuration options at build time.  Have a look at
%    <filename>pkgsrc/mk/defaults/mk.conf</filename> to get an overview
%    of what will be set there by default.  Environment variables such
%    as <varname>LOCALBASE</varname> can be set in
%    &mk.conf; to save having to remember to
%    set them each time you want to use pkgsrc.</para>
%

Occasionally, people want to ``look under the covers'' to see  what is going on
when a  package  is building  or being  installed.  This may  be for  debugging
purposes, or  out  of simple curiosity. A  number  of utility values have  been
added to help with this.

\begin{enumerate}

\item If you invoke the {\tt make} command with {\tt PKG\_DEBUG\_LEVEL=1}, then
      a huge amount of information will be displayed. For example,

\begin{verbatim}
% make patch PKG_DEBUG_LEVEL=1
\end{verbatim}

      will show all the commands that are invoked, up to and including the
      ``patch'' stage. Using {\tt PKG\_DEBUG\_LEVEL=2} will give you even
      more details.

\item If you want to know the value of a certain {\tt make} definition, then
   the {\tt VARNAME} variable   should be used,  in  conjunction with the  {\tt
   show-var} target.  e.g.  to show the  expansion  of the  {\tt make} variable
   {\tt LOCALBASE}:

\begin{verbatim}
% make show-var VARNAME=LOCALBASE
\end{verbatim}

\end{enumerate}

%    <para>If you want to install a binary package that you've either
%    created yourself (see next section), that you put into
%    pkgsrc/packages manually or that is located on a remote FTP
%    server, you can use the "bin-install" target. This target will
%    install a binary package - if available - via &man.pkg.add.1;,
%    else do a <command>make package</command>.  The list of remote FTP
%    sites searched is kept in the variable
%    <varname>BINPKG_SITES</varname>, which defaults to
%    ftp.NetBSD.org. Any flags that should be added to &man.pkg.add.1;
%    can be put into <varname>BIN_INSTALL_FLAGS</varname>.  See
%    <filename>pkgsrc/mk/defaults/mk.conf</filename> for more
%    details.</para>


%    <para>A final word of warning: If you set up a system that has a
%    non-standard setting for <varname>LOCALBASE</varname>, be sure to
%    set that before any packages are installed, as you cannot use
%    several directories for the same purpose. Doing so will result in
%    pkgsrc not being able to properly detect your installed packages,
%    and fail miserably. Note also that precompiled binary packages are
%    usually built with the default <varname>LOCALBASE</varname> of
%    <filename>/usr/pkg</filename>, and that you should
%    <emphasis>not</emphasis> install any if you use a non-standard
%    <varname>LOCALBASE</varname>.</para>


\subsection{Installing binary packages} % ----------------------------------

At the moment, installing binary packages is not documented.


\subsection{Removing packages} % -------------------------------------------

To deinstall a package, it does not matter whether it was installed from source
code or  from a  binary package.  The  {\tt robotpkg\_delete} command  does not
know it  anyway.  To delete a  package, you can just  run {\tt robotpkg\_delete
<package-name>}.  The package name can be given with or without version number.
Wildcards can  also be used  to deinstall a  set of packages, for  example {\tt
*genom*} all  packages whose  name contain  the word {\tt  genom}.  Be  sure to
include them  in quotes,  so that the  shell does  not expand them  before {\tt
robotpkg\_delete} sees them.

The {\tt -r} option is very powerful: it  removes all the packages that require
the package in question and then removes the package itself. For example:

\begin{verbatim}
% robotpkg_delete -r genom
\end{verbatim}

will remove genom and all the packages that used it; this allows
upgrading the {\tt genom} package.


\subsection{Getting information about installed packages} % ----------------

The {\tt  robotpkg\_info} shows information about installed  packages or binary
package files.


\subsection{Other administrative functions} % ------------------------------

The  {\tt robotpkg\_admin}  executes  various administrative  functions on  the
package system.

% $LAAS: using.tex 2009/03/04 16:45:50 mallet $
%
% Copyright (c) 2009 LAAS/CNRS
% All rights reserved.
%
% Permission to use, copy, modify, and distribute this software for any purpose
% with or without   fee is hereby granted, provided   that the above  copyright
% notice and this permission notice appear in all copies.
%
% THE SOFTWARE IS PROVIDED "AS IS" AND THE AUTHOR DISCLAIMS ALL WARRANTIES WITH
% REGARD TO THIS  SOFTWARE INCLUDING ALL  IMPLIED WARRANTIES OF MERCHANTABILITY
% AND FITNESS. IN NO EVENT SHALL THE AUTHOR  BE LIABLE FOR ANY SPECIAL, DIRECT,
% INDIRECT, OR CONSEQUENTIAL DAMAGES OR  ANY DAMAGES WHATSOEVER RESULTING  FROM
% LOSS OF USE, DATA OR PROFITS, WHETHER IN AN ACTION OF CONTRACT, NEGLIGENCE OR
% OTHER TORTIOUS ACTION,   ARISING OUT OF OR IN    CONNECTION WITH THE USE   OR
% PERFORMANCE OF THIS SOFTWARE.
%
%                                             Anthony Mallet on Wed Mar  4 2009
%

\section{Configuring robotpkg} % -------------------------------------------

The whole \robotpkg system is configured  {\em via} a single, centralized file,
called {\tt   robotpkg.conf}  and  placed  in  the   {\tt  /opt/openrobots/etc}
directory  by default.  This location  might be redefined  during the bootstrap
phase,  see  \xref{section:bootstrapping}{Section~\ref{section:bootstrapping}}.
During    the  bootstrap, an initial configuration     file is created with the
settings you provided to {\tt bootstrap}.

The  format of  the configuration file   is that of   the usual GNU style  {\tt
Makefile}s. The whole \robotpkg configuration  is done by setting variables  in
this  file. Note that  you can  define all  kinds of  variables, and no special
error checking (for example for spelling mistakes)  takes place, so you have to
try it out to see if it works.


\subsection{Selecting build options} % -------------------------------------

Some packages have   build time options, usually   to select between  different
dependencies,  enable  optional   support for    big   dependencies or   enable
experimental features.

To see   which options, if  any, a   package supports,  and  which  options are
mutually exclusive, run {\tt make show-options}, for example:

\begin{verbatim}
Any of the following general options may be selected:
    debug   Produce debugging information for binary programs
    doc     Compile documentation material
    lex     Use lex in place of flex
    tcl     Enable support for TCL clients
    yacc    Use yacc in place of bison

These options are enabled by default:
    doc tcl

These options are currently enabled:
    doc tcl
\end{verbatim}

The following variables can be defined  in {\tt robotpkg.conf} to select which
options to enable for a package:

\begin{itemize}
   \item {\tt PKG\_DEFAULT\_OPTIONS}, which can be used to select or  disable
   options  for  all packages  that  support them,

   \item {\tt PKG\_OPTIONS.<pkgbase>}, which can  be   used  to select  or
   disable   options specifically for package {\tt pkgbase}. Options listed
   in these variables are selected, options preceded by {\tt -} are disabled.
\end{itemize}

A few examples:

\begin{verbatim}
PKG_DEFAULT_OPTIONS=    debug
PKG_OPTIONS.genom=      doc -tcl
\end{verbatim}

It is important to note  that options that were  specifically suggested by  the
package  maintainer must be  explicitely removed if you  do not wish to include
the option.  If you  are unsure you  can view the current  state with {\tt make
show-options}.

The following settings are  consulted in the order  given, and the last setting
that selects or disables an option is used:

\begin{enumerate}
   \item the default options as suggested by the package maintainer,

   \item {\tt PKG\_DEFAULT\_OPTIONS},

   \item {\tt PKG\_OPTIONS.<pkgbase>}
\end{enumerate}

For groups of mutually exclusive options, the last option selected is used, all
others are automatically  disabled.  If  an option of  the  group is explicitly
disabled, the previously selected option,  if any, is used.   It is an error if
no option from  a  required group  of  options is  selected, and  building  the
package will fail.


\subsection{General configuration variables} % -----------------------------

In  this  section,  you can   find some  variables   that apply  to  all \robotpkg
packages.

% A complete  list of the variables that  can be configured by the user
% is  available in <filename>mk/defaults/mk.conf</filename>,  together with  some
% comments that describe each variable's intent.</para>

\begin{description}
   \item[ACCEPTABLE\_LICENSES] List of acceptable licenses. Whenever you try to
   build a package  whose license is  not in this  list, you will get  an error
   message that includes instructions on how to change this variable.

   \item[DISTDIR] Where to store the downloaded copies of the original source
   distributions used for building \robotpkg packages. The default is
   {\tt \${ROBOTPKG\_DIR}/distfiles}.

   \item[PACKAGES] The top level directory for the binary packages. The default
   is  {\tt \${ROBOTPKG\_DIR}/packages}.

%   \item[PKG\_DBDIR] Where the database  about  installed packages  is  stored.
%   The default is {\tt /opt/openrobots/var/db/pkg}.

   \item[MASTER\_SITE\_BACKUP] List  of backup locations for distribution files
   if not found locally  or  in {\tt \${MASTER\_SITES}}.  The default  is\\
   {\tt http://softs.laas.fr/openrobots/robotpkg/distfiles/}.

   \item[PKG\_DEBUG\_LEVEL] The  level of debugging  output  which is displayed
   whilst making and installing the package.  The  default value for this is 0,
   which will not  display the commands as they  are executed (normal, default,
   quiet  operation); the value 1 will  display all shell commands before their
   invocation,  and  the value  2 will  display both the  shell commands before
   their invocation, and their actual execution progress with {\tt set -x}.
\end{description}


\subsection{Variables affecting the build process} % -----------------------

\begin{description}
   \item[WRKOBJDIR] The top level   directory where, if defined,  the  separate
   working directories will get created.  This is useful for building  packages
   on a different filesystem than the \robotpkg sources.

   \item[DEPENDS\_TARGET] By default,  dependencies are only installed,  and no
   binary package is  created  for them. You  can  set  this variable  to  {\tt
   package}   to   automatically  create    binary  packages   after installing
   dependencies.

   \item[LOCALBASE] Where packages will be installed. The default value is {\tt
   /opt/openrobots}.  Do not  mix     binary  packages with    different values
   of {\tt LOCALBASE}s!


%   \item[GCC\_REQUIRED]  This specifies requirements  on  the version of GCC to
%   use  when  building  packages.   This  variable  should contain   a  list of
%   constraints in the form {\tt  \{<=,<,-,>,>=\}n}. E.g.  to specifiy a minimum
%   version of 4.2  use ``{\tt >=4.2}'', or to  specifiy gcc version 4  only use
%   ``{\tt >=4 <5}''.

\end{description}


\subsection{Additional flags to the compiler} % ----------------------------

If you wish  to set compiler variables   such as {\tt CFLAGS},  {\tt CXXFLAGS},
{\tt FFLAGS} ... please make sure to use  the {\tt +=}  operator instead of the
{\tt {=}} operator:

\begin{verbatim}
CFLAGS+= -your -flags
\end{verbatim}

Using {\tt CFLAGS=} (i.e.  without the ``{\tt +}'') may  lead to  problems with
packages that need to add their own flags.

If you want  to pass flags  to the linker, both in  the configure  step and the
build step, you  can do this  in  two ways.   Either set {\tt  LDFLAGS} or {\tt
LIBS}.  The difference between  the two is that  {\tt LIBS} will be appended to
the command line, while {\tt LDFLAGS} come earlier. {\tt LDFLAGS} is pre-loaded
with rpath settings   for machines that support  it.  As with {\tt CFLAGS}  you
should use the {\tt +=} operator:

\begin{verbatim}
LDFLAGS+= -your -linkerflags
\end{verbatim}


\chapter{The robotpkg developer's guide}
\label{chapter:developer}

\chapter{The robotpkg infrastructure internals}
\label{chapter:internal}

\end{document} % -----------------------------------------------------------
